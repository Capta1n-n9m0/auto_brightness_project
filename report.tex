\documentclass{article}
\usepackage[utf8]{inputenc}

\usepackage{graphicx}
\graphicspath{{pictures/}}
\DeclareGraphicsExtensions{.pdf,.png,.jpg}

\usepackage{xcolor}
\usepackage{colortbl}

\title{BMP File Manipulation}
\author{Abbas Aliyev, Aziz Salimli }
\date{May 2022}

\begin{document}

\maketitle

\section{Objective}

Our main task is to process images in \textbf{bmp} format.
\\[1\baselineskip]
Images in \textbf{bmp} format are saved every pixel by expansion, that is, you can access every pixel. And our task is to automatically adjust the image for contrast and adjust the contrast.
\\[1\baselineskip]
\textit{The brightness} of an image is a measure of how strongly a color appears from black. Brightness ranges from 0 to 255. The higher this value, the brighter the pixel. And in order to change the brightness of the image, we must raise the brightness of each pixel.
\\[1\baselineskip]
\textit{Contrast} is the difference in brightness between different parts of an image. Contrast also takes values from 0 to 255. For example, if an image has a spread of 130 to 160, this image is considered to be an image with poor contrast.
\\[1\baselineskip]
What should we do? Let's say we have image \textbf{\textit{A}} with spread from $\forall$ x to $\forall$ y$\rightarrow$ \textbf{\textit{A}}  = $[x,y  \mid x, y \in A,x \neq 0 , x \neq y, x<y]$, we must subtract the arrival of x from everyone. Scatter identification: $[0,x-y]$. After that we have to $255/(x-y)$. And the resulting answer is multiplied by each element in $[0,x-y]$. After this is done, the pixel values of the image will range from 0 to 255.
\\[1\baselineskip]
Due to the fact that in reality the picture consists of three Red Green Blue(RGB) colors, we must do the above manipulations with each of the colors in the image.

\section{Code Explanation}
\subsection{bmp\_core.c}

We have 4 main structures BMP\_HEADER DIB\_HEADER IMAGE\_DATA BMP\_FILE. Later we will analyze them in more detail. But first, I want to note the structures that we declare, but do not create in our project (because we believe that they are not specified in our Technical Task).
\\
\subsubsection{Declare, but not used}

If less than 8 bits are specified for the color, then we will be forced to use COLOR\_TABLE, but it is not needed in our task.
\\
BIT\_MASK is defined data that is used for masking - selecting individual bits or fields from multiple bits from a binary string or number.
\\
COLOR\_PROFILE is a set of data that characterizes a color input or output device or color space according to standards promulgated by the International Color Consortium.
\\[1\baselineskip]
\colorbox{black!80}{\textcolor{orange!65}{typedef char} \textcolor{white}{BIT\_MASKS\textcolor{orange!65}{;}}}
\\
\colorbox{black!80}{\textcolor{orange!65}{typedef char} \textcolor{white}{COLOR\_TABLE\textcolor{orange!65}{;}}}
\\
\colorbox{black!80}{\textcolor{orange!65}{typedef char} \textcolor{white}{COLOR\_PROFILE\textcolor{orange!65}{;}}}
\\[1\baselineskip]

\subsubsection{BMP\_HEADER}
BMP\_HEADER - It stores some information about our file, specifically the first 2 bytes, then 4 bytes, the next 4 bytes are reserved for the application and the last 4 bytes indicate image offsets.
\\[1\baselineskip]
\colorbox{black!80}{\textcolor{orange!65}{struct} \textcolor{white}{BMP\_HEADER\_}\textcolor{orange!65}{;}}
\subsubsection{DIB\_HEADER}
DIB\_HEADER - stores information about 4 bytes header size, 4 bytes for image width, 4 for height, 2 for colour planes, 2 for pixels, and the remaining 24 are not important.
\\[1\baselineskip]
\colorbox{black!80}{\textcolor{orange!65}{struct} \textcolor{white}{DIB\_HEADER\_}\textcolor{orange!65}{;}}
\subsubsection{IMAGE\_DATA}
IMAGE\_DATA is a structure that stores the height and width of the image.
\\[1\baselineskip]
\colorbox{black!80}{\textcolor{orange!65}{struct} \textcolor{white}{IMAGE\_DATA\_}\textcolor{orange!65}{;}}
\subsubsection{BMP\_FILE}
BMP\_FILE - a structure that stores all the structures, the root, the head of our project.
\\[1\baselineskip]
\colorbox{black!80}{\textcolor{orange!65}{struct} \textcolor{white}{BMP\_FILE}\textcolor{orange!65}{;}}
\end{document}
